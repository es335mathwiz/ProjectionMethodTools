\documentclass[12pt]{article}
\usepackage{amsmath}
\begin{document}

The model we are solving can be written as:


\begin{gather*}
q_{t} +\beta_p(1 - \rho_p)q_{t + 1} + \rho_pq_{t - 1} - \sigma_pr_{t} +
     r_{ut}=0\\
 r_{t} = \max (\bar{r}, \phi_pq_{t}) \\
 r_{ut} = \rho_{ru} r_{ut - 1} + \eta \epsilon_{y,t}
\end{gather*}

Then 
\newcommand{\xtmVec}{  \begin{bmatrix}
    q_t\\r_{t}\\r_{ut}
  \end{bmatrix}
}


\begin{gather*}
  x_t=\xtmVec
\end{gather*}

We are setting the parameters as follows:

\begin{gather*}
  \beta_p = 0.99, \phi_p = 1, 
\rho_p = 0.5, \sigma_p = 1, \rho_{ru} = 0.5,
  \bar{r} = 0.02 \\
 q^L = -10, q^H = 10, 
\sigma_u = 0.02,\\
   r_u^L = -5\sigma_u/(1 - \rho_{ru})=-.16, r_u^H=  5\sigma_u/(1 - \rho_{ru})=.16,
    \eta = 1
\end{gather*}
In order to debug our code, consider employing very low order polynomials --
first order for $q$ and second order for $r_u$.

We will have the following six chebyshev polynomials for a basis

\begin{gather*}
 \begin{array}{c}
                  1. \\
                  2. \text{qq} \\
                  6.25 \text{ru} \\
                  12.5 \text{qq} \text{ru} \\
                  78.125 \text{ru}^2-1. \\
                  156.25 \text{qq} \text{ru}^2-2. \text{qq} \\
                 \end{array}
\end{gather*}

We will have the following six evaluation points for $qq=q_{t-1}\, \text{and}\, ru=r_{ut-1}$

\begin{gather*}
   \begin{array}{cc}
                   \text{qq}\to 0.353553 & \text{ru}\to 0.138564 \\
                   \text{qq}\to -0.353553 & \text{ru}\to 0.138564 \\
                   \text{qq}\to 0.353553 & \text{ru}\to 0. \\
                   \text{qq}\to -0.353553 & \text{ru}\to 0. \\
                   \text{qq}\to 0.353553 & \text{ru}\to -0.138564 \\
                   \text{qq}\to -0.353553 & \text{ru}\to -0.138564 \\
                  \end{array}
\end{gather*}

Evaluating the polynomials at each of the points produces:

\begin{gather*}
P=     \begin{array}{cccccc}
    1. & 1. & 1. & 1. & 1. & 1. \\
    0.707107 & -0.707107 & 0.707107 & -0.707107 & 0.707107 & -0.707107 \\
    0.866025 & 0.866025 & 0 & 0 & -0.866025 & -0.866025 \\
    0.612372 & -0.612372 & 0 & 0 & -0.612372 & 0.612372 \\
    0.5 & 0.5 & -1. & -1. & 0.5 & 0.5 \\
    0.353553 & -0.353553 & -0.707107 & 0.707107 & 0.353553 & -0.353553 \\
   \end{array}
\end{gather*}

In what follows, I'm trying to go step by step through the calculations 
I think are in your MATLAB code. 

Your code computes residuals for the model equations at each node using a weighted sum of the polynomials to characterize the following term from the model:
\begin{gather*}
  \left [ \beta_p(1 - \rho_p)q_{t + 1}\right ]_i = \sum_{i} A_{i} P_{ij}
\end{gather*}

We can use the chebyshev node points and the integration nodes to determine
$q_{t-1},r_{ut-1},\epsilon_{yt}$
These determine $r_{ut}$
\begin{gather*}
 r_{ut} = \rho_{ru} r_{ut - 1} + \eta \epsilon_{y,t}
\end{gather*}


\end{document}
